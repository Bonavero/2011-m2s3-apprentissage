\documentclass[french,a4paper]{article}
\usepackage[utf8]{inputenc}
\usepackage[T1]{fontenc}
\usepackage[frenchb]{babel}
\usepackage{hyperref}
\hypersetup{%
  colorlinks,%
  citecolor=black,%
  filecolor=black,%
  linkcolor=black,%
  urlcolor=black%
}

\title{-- Rapport -- \\ Ordering-Based Search: A Simple and Effective Algorithm for Learning Bayesian Networks}
\author{BONAVERO Yoann \and DUPÉRON Georges}

\begin{document}

\maketitle
\newpage
% Document ici.

Liens qui peut-être utile : 
http://asi.insa-rouen.fr/enseignants/~pleray/RB2003/2-ApprentissageStructure.pdf

\section{Résumé de l’article :}
\subsection{Quel est le problème étudié dans l’article ?}
Le problème consiste en l'apprentissage de structures de réseaux Bayesiens à partir des données.
(page 1, Absract, "début").
Le nombre de structures possibles pour un réseaux à n nœud est exponentiel.
\subsection{Quels sont les travaux relatifs au problème ? }
Un certain nombre de techniques on été misent au point pour tenter de répondre au problème...
- Ordering-based search (Méthode utilisée ici).
- wawelet-based search (à vérifier à plutôt l'air d'être un outil utilisé combiné avec une autre technique).
- Restriction de l'espaces des arbres.
- greedy search.

Toutes ces techinques ont pour but de restreindre le parcours de recherche.

\subsection{Quel est le paradigme adopté ?}
Ordering based search.
\subsection{Quels sont les principaux résultats obtenus ?}
relativement bon comparé à d'autre algorithme recents (à revoir).

\section{Apprentissage Bayesien :}
\subsection{Décrire le fonctionnement de l'apprentissage Bayésien}
(pas sûr du toutà, trouver la meilleure structure pour décrire un ensemble de données définies

\subsection{Que signifie la fonction de scoring ?}
La fonction de "scoring" est une fonction permet de noter chaques structure selon des critères, ce qui va permettre de comparer celles-ci entre elles.
La fonction de scoring se doit donc d'être rapide, et aussi d'être localement décomposable.

\section{L'algorithme de recherche}
\subsection{Décrire l'espace de recherche utilisant les ordres}


\subsection{Décrire la recherche, le caching et le pruning}


\section{Commentaires :}
\subsection{Quels sont les avantages et inconvénients de la méthode proposée (analyse détaillée demandée) ?}
\subsection{Est-il possible d'améliorer cette méthode pour apprendre des réseaux Bayésiens dont la tree-width est bornée ?}
\subsection{Est-il possible d'apprendre plus efficacement des réseaux Bayésiens dont la structure est un arbre dirigé (ou une forêt) ?}

\end{document}
