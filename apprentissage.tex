\documentclass[french,a4paper]{article}
\usepackage[utf8]{inputenc}
\usepackage[T1]{fontenc}
\usepackage[frenchb]{babel}
\usepackage{hyperref}
\hypersetup{%
  colorlinks,%
  citecolor=black,%
  filecolor=black,%
  linkcolor=black,%
  urlcolor=black%
}

\title{-- Rapport -- \\ Ordering-Based Search: A Simple and Effective Algorithm for Learning Bayesian Networks}
\author{BONAVERO Yoann \and DUPÉRON Georges}

\begin{document}

\maketitle
\newpage
% Document ici.
\section{Résumé de l’article :}
\subsection{Quel est le problème étudié dans l’article ?}
\subsection{Quels sont les travaux relatifs au problème ? }
\subsection{Quel est le paradigme adopté ?}
\subsection{Quels sont les principaux résultats obtenus ?}

\section{Apprentissage Bayesien :}
\subsection{Décrire le fonctionnement de l'apprentissage Bayésien}
\subsection{Que signifie la fonction de scoring ?}

\section{L'algorithme de recherche}
\subsection{Décrire l'espace de recherche utilisant les ordres}
\subsection{Décrire la recherche, le caching et le pruning}

\section{Commentaires :}
\subsection{Quels sont les avantages et inconvénients de la méthode proposée (analyse détaillée demandée) ?}
\subsection{Est-il possible d'améliorer cette méthode pour apprendre des réseaux Bayésiens dont la tree-width est bornée ?}
\subsection{Est-il possible d'apprendre plus efficacement des réseaux Bayésiens dont la structure est un arbre dirigé (ou une forêt) ?}


\end{document}
